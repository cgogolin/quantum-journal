\documentclass[a4paper,twocolumn,11pt]{quantumarticle}
\pdfoutput=1
\usepackage[utf8]{inputenc}
\usepackage[english]{babel}
\usepackage[T1]{fontenc}
\usepackage[numbers, sort&compress]{natbib}
\usepackage{amsmath}
\usepackage{amssymb}
\usepackage{hyperref}
\usepackage[resetlabels]{multibib}

\newcites{article}{article references}
\newcites{book}{book references}
\newcites{misc}{misc references}
\newcites{repo}{repository references}
\newcites{web}{ references}
\newcites{other}{Other references}
\newcites{tests}{Test references}

\begin{document}
\title{Template demonstrating the quantum bibstyle}

\author{David~Wierichs}
\affiliation{Institute for Theoretical Physics, University of Cologne, Germany}
\author{Johannes~Jakob~Meyer}
\affiliation{Dahlem Center for Complex Quantum Systems, Freie Universit\"{a}t Berlin, 14195 Berlin, Germany}
\affiliation{QMATH, Department of Mathematical Sciences, University of Copenhagen, 2100 Copenhagen, Denmark}

\maketitle

\section{Overview}\label{sec:overview}

We use the opportunity of talking about bibtex entries to give an overview of the available reference classes:

\begin{center}
\begin{tabular}{ll}
    \texttt{article} & Section \ref{sec:article}\\
    \texttt{book} & Section \ref{sec:book}\\
    \texttt{repository} & Section \ref{sec:repo}\\
    \texttt{website} & Section \ref{sec:web}\\
    \texttt{misc} & Section \ref{sec:misc}\\
    Other classes & Section \ref{sec:other}\\
    \hspace{0.8cm}\texttt{booklet} & Section \ref{sec:booklet}\\
    \hspace{0.8cm}\texttt{inbook} & Section \ref{sec:inbook}\\
    \hspace{0.8cm}\texttt{incollection} & Section \ref{sec:incollection}\\
    \hspace{0.8cm}\texttt{inproceedings} & Section \ref{sec:inproceedings}\\
    \hspace{0.8cm}\texttt{manual} & Section \ref{sec:manual}\\
    \hspace{0.8cm}\texttt{mastersthesis} & Section \ref{sec:mastersthesis}\\
    \hspace{0.8cm}\texttt{phdthesis} & Section \ref{sec:phdthesis}\\
    \hspace{0.8cm}\texttt{proceedings} & Section \ref{sec:proceedings}\\
    \hspace{0.8cm}\texttt{techreport} & Section \ref{sec:techreport}\\
    \hspace{0.8cm}\texttt{unpublished} & Section \ref{sec:unpublished}\\
    Mixed tests & Section \ref{sec:tests}\\
\end{tabular}
\end{center}

All of these reference classes are available in standard bibtex style files as well, with the exception of \texttt{repository} and \texttt{website}.
Of course there may be other style files supporting reference classes with the same name, but the implementation in quantum.bst will not be based on any of those.


\section{Reference class \texttt{article}}\label{sec:article}
For the \texttt{article} class, the \texttt{title} is printed in \emph{italics}. The \texttt{journal} is not reformatted, the \texttt{volume} printed in \textbf{bold font}. We also include the \texttt{pages} if present and the \texttt{year} in round brackets ().
\texttt{doi} links are always included if given, the same holds for \texttt{eprint}. Only if neither of these two fields is given do we use the \texttt{url} to provide a hyperlink to the article.
Code repositories are linked whenever provided via the \texttt{code} field, which is a non-standard field in 
quantum.bst. 

Examples:

\begin{tabular}{ccccc}
    \texttt{doi}& \texttt{eprint} & \texttt{url} & \texttt{code} & result \\
    $\checkmark$ & $\checkmark$ & $\checkmark\big / \times$ & $\checkmark$ & \citearticle{article_doi_eprint_url_code} \\
    $\checkmark$ & $\checkmark$ & $\checkmark\big / \times$ & $\times$ & \citearticle{article_doi_eprint_url} \\
    $\times$ & $\checkmark$ & $\checkmark\big / \times$ & $\checkmark$ & \citearticle{article_eprint_url_code} \\
    $\times$ & $\checkmark$ & $\checkmark\big / \times$ & $\times$ & \citearticle{article_eprint_url} \\
    $\times$ & $\times$ & $\checkmark\big / \times$ & $\checkmark$ & \citearticle{article_url_code} \\
    $\times$ & $\times$ & $\checkmark\big / \times$ & $\times$ & \citearticle{article_url} \\
\end{tabular}

Note that in particular citations via a URL alone are not recommended. If you want to cite a website or code repository, please use the respective reference classes \texttt{website} or \texttt{repository} (see below).

\bibliographystylearticle{quantum}
\bibliographyarticle{quantum_bst_demo}

\section{Reference class \texttt{book}}\label{sec:book}
For the reference class \texttt{book}, the \texttt{title}, the \texttt{year}, the \texttt{publisher} as well as \emph{either} the \texttt{author} \emph{or} the \texttt{editor} field must be given.
The \texttt{volume}, \texttt{number} and \texttt{series}, the (publisher) \texttt{address}, the \texttt{edition} as well as links in the fields \texttt{doi}, \texttt{eprint} and \texttt{url} are optional.
The order in which links are printed is the same as for \texttt{article}, Section \ref{sec:article}.
Some example \texttt{book} references are \citebook{book_author_doi_url, book_author_url_edition}.

\bibliographystylebook{quantum}
\bibliographybook{quantum_bst_demo}

\section{Reference class \texttt{repository}}\label{sec:repo}
For the custom \texttt{repository} reference class, the \texttt{author} field is used if given but is not required (in contrast to the \texttt{article} class).
If the repository address is given via \texttt{code} (strongly recommended), a properly formatted repository name is printed and links to the given address, including potentially version-, branch- or even commit-specific links.
If no \texttt{code} entry is given, \texttt{url} is used as address instead, without any formatting of the printed text; Either \texttt{code} or \texttt{url} have to be provided.
A title is not considered even if given.
TODO: Consider a year in any way?

\begin{tabular}{ccc}
    \texttt{code}& \texttt{url} & result \\
    $\checkmark$ & $\checkmark\big / \times$ &\citerepo{repo_code_url} \\
    $\times$ & $\checkmark$ &\citerepo{repo_url} \\
    $\times$ & $\times$ & invalid \\
\end{tabular}

Note that if you want both a \texttt{url} and a \texttt{code} link to be displayed, you can use the \texttt{website} reference class presented below for that.

\bibliographystylerepo{quantum}
\bibliographyrepo{quantum_bst_demo}

\section{Reference class \texttt{website}}\label{sec:web}
For the new custom reference class \texttt{website}, we require a \texttt{title} and a \texttt{url} which are both printed always.
\texttt{author} is optional and printed if given, the same holds for \texttt{code}, which is formatted as repository link like for \texttt{repository}. If you want to provide \texttt{code} but not \texttt{url}, the reference class \texttt{repository} (see above) is made for you.

\begin{tabular}{ccc}
    \texttt{author} & \texttt{code} & result \\
    $\checkmark$ & $\checkmark$ &\citeweb{web_author_code} \\
    $\times$ & $\checkmark$ &\citeweb{web_code} \\
    $\checkmark$ & $\times$ &\citeweb{web_author} \\
    $\times$ & $\times$ &\citeweb{web} \\
\end{tabular}

Note that if you want both a \texttt{url} and a \texttt{code} link to be displayed, you can use the \texttt{website} reference class presented below for that.

\bibliographystyleweb{quantum}
\bibliographyweb{quantum_bst_demo}

\section{Reference class \texttt{misc}}\label{sec:misc}
The reference class \texttt{misc} is meant to be used for miscellaneous entries that do not fall into any of the provided categories.
As such, \texttt{misc} entries display the generic properties \texttt{author}, \texttt{title}, \texttt{howpublished}, \texttt{date}, \texttt{eprint} and \texttt{note}, the only requirement being at least one of these fields to be provided and non-empty.
As the reference class \texttt{article} covers the case of preprint articles, the \texttt{misc} class was modified to refer back to \texttt{article} if \texttt{archivePrefix} is set to ``arxiv'' or an anyhow capitalized version thereof \emph{and} \texttt{primaryClass} is provided and non-empty.

We provide some examples, not covering all cases, because \texttt{misc} is very flexible and there are many possibilities.
\begin{itemize}
    \item A citation that actually is an article on the arXiv: \citemisc{misc_arxiv}
    \item A footnote-like reference only containing a note: \citemisc{misc_note}
    \item A reference to a private correspondence: \citemisc{misc_correspondence}
\end{itemize}

\bibliographystylemisc{quantum}
\bibliographymisc{quantum_bst_demo}

%\section{Other reference classes}\label{sec:other}
%\subsection{Reference class \texttt{booklet}}\label{sec:booklet}

%\citeother{article_url_code}

\subsection{Reference class \texttt{inbook}}\label{sec:inbook}

The reference class \texttt{inbook} is an alias for \texttt{book} with the additional requirement that \texttt{chapter}, \texttt{pages} or both are provided.
Examples would be Refs.~\citeother{inbook_author_doi_url_chapter, inbook_author_doi_url_chapter_pages, inbook_author_doi_url_volume_chapter_pages}.

%\subsection{Reference class \texttt{incollection}}\label{sec:incollection}

%\citeother{article_url_code}

%\subsection{Reference class \texttt{inproceedings}}\label{sec:inproceedings}

%\citeother{article_url_code}

%\subsection{Reference class \texttt{manual}}\label{sec:manual}

%\citeother{article_url_code}

%\subsection{Reference classes \texttt{mastersthesis}}\label{sec:mastersthesis}

%\citeother{article_url_code}

%\subsection{Reference classes \texttt{phdthesis}}\label{sec:phdthesis}

%\citeother{article_url_code}

%\subsection{Reference class \texttt{proceedings}}\label{sec:proceedings}

%\citeother{article_url_code}

%\subsection{Reference class \texttt{techreport}}\label{sec:techreport}

%\citeother{article_url_code}

%\subsection{Reference class \texttt{unpublished}}\label{sec:unpublished}

%\citeother{article_url_code}

\bibliographystyleother{quantum}
\bibliographyother{quantum_bst_demo}

\section{Tests}\label{sec:tests}
Directly from the arxiv~\citetests{hubregtsen2021training}, arxiv via Zotero~\citetests{hubregtsen2021training_2}, some more testcases~\citetests{holevo2012quantum,Holevo_2012,akers2020simple,katariya2021geometric,katariya2021geometric_2,liang2019fisher-rao,jain2010guaranteed}

\bibliographystyletests{quantum}
\bibliographytests{quantum_bst_demo}
	
\end{document}
