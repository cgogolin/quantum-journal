\documentclass[a4paper,twocolumn,11pt]{quantumarticle}
\pdfoutput=1
\usepackage[utf8]{inputenc}
\usepackage[english]{babel}
\usepackage[T1]{fontenc}
\usepackage{amsmath}
\usepackage{amssymb}
\usepackage{hyperref}
\usepackage[resetlabels]{multibib}
\newcites{article}{Article references}
\newcites{repo}{Repository references}
\newcites{web}{Website references}
\newcites{book}{Book references}

\begin{document}
\title{Template demonstrating the quantum bibstyle}

\author{David Wierichs}
\affiliation{Institute for Theoretical Physics, University of Cologne, Germany}

\maketitle

\onecolumn
\section{Reference class <article>}
For the <article> class, the <title> is printed in \emph{italics}. The <journal> is not reformatted, the <volume> printed in \textbf{bold font}. We also include the <pages> if present and the <year> in round brackets ().
<doi> links are always included if given, the same holds for <eprint>. Only if neither of these two fields is given do we use the <url> to provide a hyperlink to the article.
Code repositories are linked whenever provided via the <code> field, which is a non-standard field in 
quantum.bst. 

Examples:

\begin{tabular}{ccccc}
    <doi>& <eprint> & <url> & <code> & result \\
    $\checkmark$ & $\checkmark$ & $\checkmark\big / \times$ & $\checkmark$ & \citearticle{article_doi_eprint_url_code} \\
    $\checkmark$ & $\checkmark$ & $\checkmark\big / \times$ & $\times$ & \citearticle{article_doi_eprint_url} \\
    $\times$ & $\checkmark$ & $\checkmark\big / \times$ & $\checkmark$ & \citearticle{article_eprint_url_code} \\
    $\times$ & $\checkmark$ & $\checkmark\big / \times$ & $\times$ & \citearticle{article_eprint_url} \\
    $\times$ & $\times$ & $\checkmark\big / \times$ & $\checkmark$ & \citearticle{article_url_code} \\
    $\times$ & $\times$ & $\checkmark\big / \times$ & $\times$ & \citearticle{article_url} \\
\end{tabular}

Note that in particular citations via a URL alone are not recommended. If you want to cite a website or code repository, please use the respective reference classes <website> or <repository> (see below).

\bibliographystylearticle{quantum.bst}
\bibliographyarticle{quantum_bst_demo.bib}

\pagebreak
\section{Reference class <repository>}
For the custom <repository> reference class, the <author> field is used if given but is not required (in contrast to the <article> class).
If the repository address is given via <code> (strongly recommended), a properly formatted repository name is printed and links to the given address, including potentially version-, branch- or even commit-specific links.
If no <code> entry is given, <url> is used as address instead, without any formatting of the printed text; Either <code> or <url> have to be provided.
A title is not considered even if given.
TODO: Consider a year in any way?

\begin{tabular}{ccc}
    <code>& <url> & result \\
    $\checkmark$ & $\checkmark\big / \times$ &\citerepo{repo_code_url} \\
    $\times$ & $\checkmark$ &\citerepo{repo_url} \\
    $\times$ & $\times$ & invalid \\
\end{tabular}

Note that if you want both a <url> and a <code> link to be displayed, you can use the <website> reference class presented below for that.

\bibliographystylerepo{../bibtex_wand/quantum2}
\bibliographyrepo{quantum_bst_demo.bib}

\section{Reference class <website>}
For the new custom reference class <website>, we require a <title> and a <url> which are both printed always.
<author> is optional and printed if given, the same holds for <code>, which is formatted as repository link like for <repository>. If you want to provide <code> but not <url>, the reference class <repository> (see above) is made for you.

\begin{tabular}{ccc}
    <author>& <code> & result \\
    $\checkmark$ & $\checkmark$ &\citeweb{web_author_code} \\
    $\times$ & $\checkmark$ &\citeweb{web_code} \\
    $\checkmark$ & $\times$ &\citeweb{web_author} \\
    $\times$ & $\times$ &\citeweb{web} \\
\end{tabular}

Note that if you want both a <url> and a <code> link to be displayed, you can use the <website> reference class presented below for that.

\bibliographystyleweb{../bibtex_wand/quantum2}
\bibliographyweb{quantum_bst_demo.bib}

\end{document}
